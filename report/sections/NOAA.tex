As an extra step, we consider the Nebraska weather prediction data provided by the U.S. National Oceanic and Atmospheric Administration\footnote{\url{ftp://ftp.ncdc.noaa.gov/pub/data/gsod/}}. This a non-stationary dataset which exhibits periodic concept drift, meaning that the distribution of \(p(y|X)\) changes over time. This poses certain challenges for all classical learning methods which assume that the distribution of data never changes, i.e., data is always stationary. Several research works on concept drift detection and adaptation, such as \cite{elwellIncrementalLearningConcept2011}, have experimented on the dataset as a benchmark. We use the preprocessed version provided by the authors in \cite{elwellIncrementalLearningConcept2011}\footnote{\url{http://users.rowan.edu/~polikar/nse.html}}. The preprocessed version considers a binary classification problem (rain vs. no-rain). We compare and report the performance of the classification methods on the entire dataset.